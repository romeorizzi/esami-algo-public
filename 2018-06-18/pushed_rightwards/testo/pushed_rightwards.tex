\renewcommand{\nomebreve}{pushed\_rightwards}
\renewcommand{\titolo}{Pushed towards the right border}

\introduzione{}

Visiteremo alcune celle di un array raccogliendo da ciascuna di esse il numero di gemme ivi contenute.
Le celle sono numerate da $1$ ad $n$, da sinistra verso destra,
e la cella dalla quale partiamo è sempre la cella~$1$.
Per ogni $i=1,2,\ldots, n$, la cella~$i$ contiene $g_i$ gemme, che avremo modo di raccogliere se visiteremo effettivamente tale cella.
In ogni cella è presente anche un troll, col troll nella cella~$i$
temibile di livello $t_i$. Quando sono nella cella~$i$ faccio appena in tempo a raccogliere le gemme e debbo subito scappare in una cella $j$
con $j > i + t_i$.

Si pianifichi quali celle dell'array visitare per massimizzare il numero di gemme raccolte.\\


\sezionetesto{Dati di input}
La prima riga del file \verb'input.txt' contiene un numero intero e positivo $n$, la lunghezza dell'array.
La seconda riga contiene una sequenza di $n$ numeri naturali separati da spazio:
l'$i$-esimo di questi numeri è $g_i$, il numero di gemme nella cella~$i$.
La terza riga contiene una sequenza di $n$ numeri naturali separati da spazio:
l'$i$-esimo di questi numeri è $t_i$, la temibilità del troll della cella~$i$.

\sezionetesto{Dati di output}
Nella prima ed unica riga del file \verb'output.txt'
si scriva il massimo numero di gemme che risulta possibile raccogliere.\\


% Esempi
\sezionetesto{Esempio di input/output}
\esempio{
6

0 1 3 3 0 0

0 2 1 6 0 9
}{3}
\esempio{
6

0 0 2 3 1 1

0 0 1 2 0 5
}{4}

\section*{Subtask}

  \begin{itemize}
    \item \textbf{Subtask 1 [0 punti]:} i due esempi del testo.
    \item \textbf{Subtask 2 [20 punti]:} $n \leq 10$.
    \item \textbf{Subtask 3 [30 punti]:} $t_i = 1$ per ogni cella~$i$, $n \leq 1\,000$.
    \item \textbf{Subtask 5 [20 punti]:} $n \leq 1\,000$.
    \item \textbf{Subtask 6 [30 punti]:} $n \leq 100\,000$.
  \end{itemize}
  
