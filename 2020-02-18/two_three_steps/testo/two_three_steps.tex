\renewcommand{\nomebreve}{two\_three\_steps}
\renewcommand{\titolo}{Better two or three steps?}

\introduzione{}

Nella nostra passeggiata lungo il corridoio del reparto geriatria,
facciamo pausa ogni 2 oppure 3 piastrelle, e lì ripeschiamo un ricordo felice dalla nostra memoria, aiutati dalle foto di paesaggi appese per nascondere la realtà delle misere pareti.
E però il valore di quel ricordo ci ostiniamo a voler credere sia dettato dalla fotografia, non da noi stessi, e potremmo essere indotti a ritenere di voler programmare le nostre pause in un corridoio rappresentabile come un array di numeri naturali, che noi si percorre avanzando di due o tre posizioni per volta, e aggiungendo il numero ivi riscontrato, ad ogni sosta, al senso ultimo della nostra vita di follia.


  Si massimizzi il senso di tutto questo.\\


\sezionetesto{Dati di input}
La prima riga del file \verb'input.txt' contiene un numero intero e positivo $n$, il numero di foto appese lungo il corridoio.
La seconda riga offre una sequenza di $n$ numeri naturali separati da spazio:
l'$i$-esimo di questi numeri, $GGG_i$, esprime la qualità del sogno ispiratomi da quella fotografia. E ovviamente i numeri sono sommabili, no?

\sezionetesto{Dati di output}
Nella prima ed unica riga del file \verb'output.txt'
si scriva il massimo valore per la somma delle qualità dei sogni di cui fruirò.\\


% Esempi
\sezionetesto{Esempio di input/output}
\esempio{
6

0 1 3 3 0 0
}{3}
\esempio{
6

0 0 2 3 1 1
}{4}

\section*{Subtask}

  \begin{itemize}
    \item \textbf{Subtask 1 [0 punti]:} i due esempi del testo.
    \item \textbf{Subtask 2 [25 punti]:} $n \leq 10$.
    \item \textbf{Subtask 3 [25 punti]:} $n \leq 100$.
    \item \textbf{Subtask 4 [30 punti]:} $n \leq 1\,000$.
    \item \textbf{Subtask 5 [20 punti]:} $n \leq 1\,000\,000$.
  \end{itemize}
  
