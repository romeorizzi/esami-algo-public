% Template per generare 

\documentclass[a4paper,11pt]{article}
\usepackage{lmodern}
\renewcommand*\familydefault{\sfdefault}
\usepackage{sfmath}
\usepackage[utf8]{inputenc}
\usepackage[T1]{fontenc}
\usepackage[italian]{babel}
\usepackage{indentfirst}
\usepackage{graphicx}
\usepackage{tikz}
\usepackage{listings}
\newcommand*\circled[1]{\tikz[baseline=(char.base)]{
		\node[shape=circle,draw,inner sep=2pt] (char) {#1};}}
\usepackage{enumitem}
% \usepackage[group-separator={\,}]{siunitx}
\usepackage[left=2cm, right=2cm, bottom=3cm]{geometry}
\frenchspacing

\newcommand{\num}[1]{#1}

% Macro varie...
\newcommand{\file}[1]{\texttt{#1}}
\renewcommand{\arraystretch}{1.3}
\newcommand{\esempioA}[2]{
\noindent\begin{minipage}{\textwidth}
\begin{tabular}{|p{7cm}|p{9cm}|}
	\hline
      \textbf{\file{input (da stdin)}} & \textbf{\file{output (su stdout)}}\\
	\hline
	\tt \small #1 &
	\tt \small #2 \\
	\hline
\end{tabular}
\end{minipage}
}
\newcommand{\esempioB}[2]{
\noindent\begin{minipage}{\textwidth}
\begin{tabular}{|p{9cm}|p{7cm}|}
	\hline
      \textbf{\file{input (da stdin)}} & \textbf{\file{output (su stdout)}}\\
	\hline
	\tt \small #1 &
	\tt \small #2 \\
	\hline
\end{tabular}
\end{minipage}
}

% Dati del task
\newcommand{\gara}{Esame Algoritmi 2020-02-18}
\newcommand{\nome}{Verifica offline di minimo antenato comune}
\newcommand{\nomebreve}{offline\_LCA}

\begin{document}
% Intestazione
\noindent{\Large \gara}
\vspace{0.5cm}

\noindent{\Huge \textbf \nome~(\texttt{\nomebreve})}

% Descrizione breve del problema
\section*{Descrizione breve del problema}

Lavoriamo con alberi orientati (dove da ogni nodo è presente un unico percorso verso la radice). Una foresta è un insieme di tali alberi senza nodi in comune.
Il tuo primo obiettivo sarà stabilire quanti siano gli alberi nella foresta. Ti verrà chiesto poi, come secondo livello di difficoltà, di decidere se un nodo è antenato di un altro. Infine, per un insieme di triple $(u,v,w)$ assegnate a tè tutte in bella partenza, dovrai stabilire quando $w$ sia il minimo antenato comune (lowest common ancestor) tra $u$ e $v$.

% Maggiori dettagli
\section*{Maggiori dettagli}

Ricevi in input una foresta i cui $N$ nodi sono numerati da $0$ ad $N-1$.
Ogni nodo $v$ dichiara il padre $p[v]$, se ne ha uno;
in caso contrario $p[v]=-1$.
Quando $p[v]\neq -1$ l'arco $(v,p[v])$ è presente nella foresta e $v$ e $p[v]$ appartengono sicuramente allo stesso albero della stessa.
Un nodo $x$ è antenato di $v$ se $x=v$ oppure se $p[v]\neq -1$ e $x$ è antenato di $p[v]$. In formula, $A(v) := \{v\} \cup A(p[v])$, dove $A(-1) = \emptyset$.
Questa formula definisce l'insieme degli antenati (ancestors) di ogni nodo, in quanto viene garantito che la relazione ``padre di'' sia aciclica ossia che la sua chiusura transitiva non contenga la coppia $(v,v)$ per nessun nodo $v$.

Denotiamo con $CA(u,v) := A(u)\cap A(v)$ l'insieme degli antenati comuni (common ancestors) di $u$ e di $v$.
La profondità (depth) di un nodo $v$ è data da $d(v) := |A(v)|$.
Il minimo antenato comune (lowest common ancestors) di $u$ e $v$
è definito da $LCA(u,v) := \arg\max_{w\in CA(u,v)} d(w)$ quando $u$ e $v$ appartengono ad uno stesso albero della foresta; altrimenti poniamo $LCA(u,v) := -1$.

% Input
\section*{Dati di input}

La prima riga contiene due numeri naturali separati da spazio:
il numero di nodi $N$ ed il numero di query $Q$.
La seconda riga contiene $N$ numeri interi nell'intervallo $[-1,N-1]$, separati da spazio.
L'$i$-esimo di questi numeri è il nome di $p[i-1]$, oppure $-1$ se il nodo $i-1$ non ha padre.
Seguono $Q$ righe, ciascuna specifica una diversa query cui devo rispondere, e posso leggerle tutte prima di cominciare a fornire delle risposte.
Ciascuna query è specificata fornendo tre numeri, $u$, $v$ e $w$ separati da spazio. La domanda posta con la query $Q(u,v,w)$ è se valga o meno che $w=LCA(u,v)$.
Questo formato consente di mettere a cappello comune, come caso particolare, alcuni tipi di domande più semplici:
\begin{description}
  \item[$Q(u,v,w=-1)$] per chiedere se $u$ e $v$ siano in alberi diversi;
  \item[$Q(u,v,w=v)$] per chiedere se $v$ sia un antenato di $u$.
\end{description}


% Output
\section*{Dati di output}

L'output deve avvenire su stdout.
Nella prima riga si scriva il numero di alberi che compongono la foresta.
Seguono $Q$ righe, dove nella $i$-esima di queste righe si scrive la risposta alla query $i$-esima nell'input: $1$ se $w=LCA(u,v)$ e $0$ altrimenti.

% Esempi
\section*{Esempio di input/output}

\esempioB{10 7

7 6 7 0 2 0 -1 -1 1 2

3 4 -1

3 1 -1

5 7 7

7 5 5

4 9 0

4 9 7

4 9 2
}{2

0

1

1

0

0

0

1}

{\bf Spiegazione:}
La foresta ricevuta in input è la sequente:
\begin{verbatim}
          7       6
         / \      |
        0   2     1
       / \  |\    |
      3   5 4 9   8
\end{verbatim}
Il primo $2$ nell'output dice che la foresta consta di due alberi. Ed ecco le motivazioni delle seguenti risposte:\\
$Q(3,4,-1)=0$ perchè i nodi $3$ e $4$ sono ricompresi nello stesso albero.\\
$Q(3 1 -1)=1$ perchè i nodi $3$ e $1$ non sono ricompresi nello stesso albero.\\
$Q(5,7,7)=1$ perchè $5$ è un discendente di $7$.\\ 
$Q(7,5,5)=0$ perchè $7$ non è un discendente di $5$.\\
$Q(4,9,0)=0$ perchè $4$ non è un discendente di $0$ (e per altro nemmeno $9$ lo è).\\
$Q(4,9,7)=0$ perchè, anche se $4$ e $9$ sono entrambi discendenti di $7$, tuttavia sono anche entrambi discendenti di $2$, che è un discendente stretto di $7$.\\
$Q(4,9,2)=1$ perchè $2$ è quel discendente comune a $4$ e $9$ che di più sotto non ce ne è.\\ 



%Assunzioni
%\section*{Assunzioni}
%\begin{itemize}[nolistsep, noitemsep]
%\item dove $n$ \`e il numero di nodi dell'albero,
%      vale sempre che $1 \le n \le 300\,000 $;
%\item tempo limite: un secondo.
%\end{itemize}

% Subtasks
\section*{Subtask}

\begin{itemize}
\item \textbf{Subtask 1 [0 punti]:} i casi di esempio forniti alla pagina del problema, essi includono i tre casi sopra.
\item \textbf{Subtask 2 [5 punti]:} contare gli alberi: $Q=0$, $N \le 500$.
\item \textbf{Subtask 3 [7 punti]:} contare gli alberi: $Q=0$, $N \le 300\,000$.
\item \textbf{Subtask 4 [4 punti]:} stabilire se $u$ e $v$ in alberi diversi ($w=-1$ in ogni query), $Q,N \le 500$.
\item \textbf{Subtask 5 [10 punti]:} stabilire se $u$ e $v$ in alberi diversi ($w=-1$ in ogni query), $Q,N \le 200\,000$.
\item \textbf{Subtask 6 [7 punti]:} stabilire se $v$ è ancestor di $u$ ($w=v$ in ogni query), $Q,N \le 1\,000$.
\item \textbf{Subtask 7 [15 punti]:} stabilire se $v$ è ancestor di $u$ ($w=v$ in ogni query), $Q,N \le 100\,000$.
\item \textbf{Subtask 8 [12 punti]:} verifica offline degli LCA, $Q,N \le 500$.
\item \textbf{Subtask 9 [40 punti]:} verifica offline degli LCA, $Q,N \le 100\,000$.
\end{itemize}


\end{document}
