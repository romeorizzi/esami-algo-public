\documentclass[a4paper,11pt]{article}
\usepackage{nopageno} % visto che in questo caso abbiamo una pagina sola
\usepackage{lmodern}
\renewcommand*\familydefault{\sfdefault}
\usepackage{sfmath}
\usepackage[utf8]{inputenc}
\usepackage[T1]{fontenc}
\usepackage[italian]{babel}
\usepackage{indentfirst}
\usepackage{graphicx}
\usepackage{tikz}
\usepackage{wrapfig}
\newcommand*\circled[1]{\tikz[baseline=(char.base)]{
		\node[shape=circle,draw,inner sep=2pt] (char) {#1};}}
\usepackage{enumitem}
% \usepackage[group-separator={\,}]{siunitx}
\usepackage[left=2cm, right=2cm, bottom=2cm]{geometry}
\frenchspacing

\newcommand{\num}[1]{#1}

% Macro varie...
\newcommand{\file}[1]{\texttt{#1}}
\renewcommand{\arraystretch}{1.3}
\newcommand{\esempio}[2]{
\noindent\begin{minipage}{\textwidth}
\begin{tabular}{|p{11cm}|p{5cm}|}
	\hline
	\textbf{File \file{input.txt}} & \textbf{File \file{output.txt}}\\
	\hline
	\tt \small #1 &
	\tt \small #2 \\
	\hline
\end{tabular}
\end{minipage}
}

\newcommand{\sezionetesto}[1]{
    \section*{#1}
}

\newcommand{\gara}{Esame algoritmi 2018-07-25 VR}

%%%%% I seguenti campi verranno sovrascritti dall'\include{nomebreve} %%%%%
\newcommand{\nomebreve}{}
\newcommand{\titolo}{}

% Modificare a proprio piacimento:
\newcommand{\introduzione}{
%    \noindent{\Large \gara{}}
%    \vspace{0.5cm}
    \noindent{\Huge \textbf \titolo{}~(\texttt{\nomebreve{}})}
    \vspace{0.2cm}\\
}

\begin{document}

\renewcommand{\nomebreve}{count\_poldo}
\renewcommand{\titolo}{Contare le sottosequenze crescenti}

\introduzione{}

Si riceve in input una sequenza $S$ di $N$ numeri interi $x_0,\dots,x_{N-1}$.
Contare quante sono le sottosequenze non vuote di $S$ strettamente crescenti.

Si chiede di restituire il numero di tali sottosequenze modulo $1024$.

Una sottosequenza di $S$ si ottiene quando si selezionino alcuni degli elementi di $S$ come elementi da tenere, e si nascondano gli altri elementi. La sottosequenza è essa stessa una sequenza in quanto viene mantenuto l'ordine tra gli elementi conservati. La sottosequenza è non vuota se almeno un elemento è stato selezionato e mantenuto. La sottosequenza è identificata dagli indici degli elementi selezionati, non dal loro valore. (Vedi esempi.) Pertanto, due diverse sottosequenze possono risultare uguali quando viste come sequenze, ma noi le dobbiamo conteggiare entrambe. 

\section*{Dati di input}
La prima riga del file \verb'input.txt' contiene il numero $N$: la lunghezza della sequenza.
Le successive $N$ righe contengono, ciascuna, un numero della sequenza.
L'$i$-esimo elemento della sequenza è quello nella riga $i+1$.

\section*{Dati di output}
Nel file \verb'output.txt' si scriva un unica riga contenente
il numero delle sottosequenze non vuote e crescenti, modulo $1024$.

% Esempi
\sezionetesto{Esempio di input/output}
\esempio{
4

5

1

1

3
}{6}

La sequenza $5, 1, 1, 3$ ammette $6$ sottosequenze crescenti non vuote. Esse sono:
\begin{enumerate}
  \item[$S_1 =$] $\underline{\mathbf{5}}, 1, 1, 3$
  \item[$S_2 =$] $5, \underline{\mathbf{1}}, 1, 3$
  \item[$S_3 =$] $5, \underline{\mathbf{1}}, 1, \underline{\mathbf{3}}$
  \item[$S_4 =$] $5, 1, \underline{\mathbf{1}}, 3$
  \item[$S_5 =$] $5, 1, \underline{\mathbf{1}}, \underline{\mathbf{3}}$
  \item[$S_6 =$] $5, 1, 1, \underline{\mathbf{3}}$
\end{enumerate}

Si noti come $S_3$ ed $S_5$ risultino eguali quando considerate come sequenze.
(Lo stesso vale anche per $S_2$ ed $S_4$).

\esempio{
11

100

1

2

3

4

5

6

7

8

9

10
}{0}

Ci sono esattamente 1024 sottosequenze crescenti.
Modulo 1024, la risposta corretta è perciò 0.

% Assunzioni
\sezionetesto{Assunzioni e note}
\begin{itemize}[nolistsep, noitemsep]
  \item $1 \leq N \leq 100\,000$.
  \item $1 \leq x_i \leq 1\,000\,000$.
\end{itemize}
  
\section*{Subtask}
\begin{itemize}
  \item \textbf{Subtask 1 [0 punti]:} gli esempi del testo.
  \item \textbf{Subtask 2 [10 punti]:} $N \leq 20$, la sequenza è una permutazione di $0, \ldots, N-1$.
  \item \textbf{Subtask 3 [10 punti]:} $N \leq 20$.
  \item \textbf{Subtask 4 [10 punti]:} $N \leq 500$, la sequenza è una permutazione di $0, \ldots, N-1$.
  \item \textbf{Subtask 5 [10 punti]:} $N \leq 5\,000$, la sequenza è una permutazione di $0, \ldots, N-1$.
  \item \textbf{Subtask 6 [10 punti]:} $N \leq 5\,000$, gli elementi della sequenza sono tutti distinti.
  \item \textbf{Subtask 7 [20 punti]:} $N \leq 100\,000$, $0 \leq x_i \leq 10$.
  \item \textbf{Subtask 8 [20 punti]:} $N \leq 100\,000$, la sequenza è una permutazione di $0, \ldots, N-1$.
  \item \textbf{Subtask 9 [10 punti]:} $N \leq 100\,000$, nessuna restrizione.
\end{itemize}


\end{document}
