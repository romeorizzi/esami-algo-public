\renewcommand{\nomebreve}{linear\_diff\_constraints}
\renewcommand{\titolo}{Mal Comune Mezzo Gaudio - {\large ripreso dalle OII 2019}\\}

\introduzione{}

Al fine di abbellire la nostra città, le $N$ famiglie più potenti della stessa costruiranno ciascuna una diversa torre, che dimostri la magnificenza sia del borgo che del casato. Poichè non ovunque il terreno è stabile, per ogni torre $i$, con $i\in [0,N)$, è stata prescritta un'altezza massima $H[i]$. Ciò nonostante tutti hanno ambizione, se non a costruire la torre più alta, quantomeno a non sfigurare rispetto ad altri in modo impropio.
Sulle altezze effettive $h$ delle varie torri si sono così consolidati $M$ ulteriori vincoli della forma $h[i_1] \geq h[i_2] - \delta$, con $\delta \in  \mathbf{N}$. Più precisamente, per ogni $j=0,\ldots,M-1$, si avrà il vincolo:
\[
   h[B_j] \leq h[A_j] + C_j 
\]
dove $A_j, B_j$ ed anche $C_j$ sono sempre numeri interi non-negativi, e con $A_j, B_j < N$.

Calcola il massimo valore di $\sum_{i=0}^{N-1} h[i]$ su tutti i possibili piani regolatori $h:[0,N)\mapsto \mathbf{N}$ che non violino nè i tetti sulle altezze massime (ossia con $h\leq H$)  nè alcuno degli $M$ vincoli di confronto. 


\sezionetesto{Input ed Output}

Per sgravarti dalla scrittura di codice per l'input e l'output trovi dei grader e dei template tra gli attachment del problema (quella che contiene il testo del problema e che offre alcune istanze tra i suoi attachments). L'utilizzo dei grader è esemplificato dai template scaricabili. Per risolvere il problema puoi limitarti ad integrare il teplate per il linguaggio scelto e sottometterlo. Sono supportati tutti e tre i linguaggi \verb'C', \verb'c++' e \verb'python3'.
Internamente ai grader, input ed output avvengono da \verb'stdin' e su \verb'stdout' rispettivamente. Alcune coppie input/output corrette, salvate su file, sono fornite sempre alla pagina del problema.
Grazie ai grader, nella scrittura del codice potrai concentrarti sulla sola parte di soluzione del problema. Ad esempio, nel caso di \verb'C', \verb'c++' dovrai implementare la seguente funzione:

\begin{center}\begin{tabularx}{\textwidth}{|c|X|}
\hline
C  & \verb|long long costruisci(int N, int M, long long* H, int* A, int* B, int* C);|\\
C++ & \verb|long long costruisci(int N, int M, vector<long long>& H, vector<int>& A,|\\
& \hspace{4.25cm}\verb|vector<int>& B, vector<int>& C);|\\
\hline
\end{tabularx}\end{center}

\begin{itemize}[nolistsep]
	\item L'intero $N$ che rappresenta il numero di torri da costruire.
	\item L'intero $M$ che rappresenta il numero di vincoli imposti dai
	      costruttori.
	\item Il vettore $H$, indicizzato da $0$ a $N-1$, che contiene le altezze
	      massime di ciascun grattacielo.
	\item I vettori $A$, $B$ e $C$, indicizzati da $0$ a $M-1$, che contengono i
	      vincoli dei costruttori.
\end{itemize}

\medskip

La funzione \texttt{costruisci} deve restituire la massima altezza totale delle torri che è possibile ottenere rispettando tutti i vincoli.

Anche nel caso di \verb'python' puoi limitarti a modificare solo il template, inserendo la soluzione dove segnalato, dando per scontato che le variabili $N$, $M$, $H$, $A$, $B$ e $C$ siano già state inizializzate correttamente. Il tuo compito sarà caricare la risposta nella variabile `\verb'answer'`. Nel caso di python ti resta la libertà di scrivere un codice che gestisca lui anche gli standard input e standard output, dato che quì è il codice da tè sottomesso a chiamare il grader invece che il contrario come in \verb'C', \verb'c++'.


\newpage
\sezionetesto{Formati dei file Input e Output}

Il file di input è composto $M+2$ righe, contenenti:
\begin{itemize}[nolistsep,itemsep=2mm]
\item Riga $1$: gli interi $N$ e $M$.
\item Riga $2$: gli $N$ interi $H_0, \dots, H_{N-1}$.
\item Righe $3, \dots, M+2$: i tre interi $A_j$, $B_j$ e $C_j$ che rappresentano il $j$-esimo vincolo.
\end{itemize}

Il file di output consiste di una sola riga contenente il valore numerico corretto per la massima possibile somma delle altezze.





% Esempi
\sezionetesto{Esempi di input/output}

In attachment alla pagina del problema trovate diverse coppie input/output tra cui le seguenti.


\vspace{0.5cm}
\esempio{4 5

2 3 6 3

0 1 4

1 2 1

2 0 1

0 3 0

3 2 2}{11}

In questo primo esempio abbiamo $4$ torri le cui altezze possiamo indicare con $h[0], h[1], h[2], h[3]$. Su queste altezze vengono forniti $5$ vincoli lineari alle differenze, che sono:\\
\indent 1) riga~3: {\tt 0 1 4} $\rightarrow$ $h[1] \leq h[0] + 4$\\ 
\indent 2) riga~4: {\tt 1 2 1} $\rightarrow$ $h[2] \leq h[1] + 1$\\
\indent 3) riga~5: {\tt 2 0 1} $\rightarrow$ $h[0] \leq h[2] + 1$\\
\indent 4) riga~6: {\tt 0 3 0} $\rightarrow$ $h[3] \leq h[0] + 0$\\
\indent 5) riga~7: {\tt 3 2 2} $\rightarrow$ $h[2] \leq h[3] + 2$\\
La seconda riga codifica invece le limitazioni superiori:
\[
h[0] \leq 2, \quad h[1] \leq 3, \quad h[2] \leq 6, \quad h[3] \leq 3
\]
La risposta corretta $11 = 2+3+4+2$ è il valore di una soluzione ottima del problema di ottimizzazione $\max h[0]+h[1]+h[2]+h[3]$ soggetto alle limitazioni superiori e ai 5 vincoli riportati sopra. Dal vincolo~2 $h[2] \leq h[1] + 1$ e dal tetto $h[1] \leq 3$ possiamo infatti dedurre $h[2] \leq 4$.
Similmente, dal vincolo~4 $h[3] \leq h[0] + 0$ e dal tetto $h[0] \leq 2$ segue $h[3] \leq 2$.



\vspace{0.5cm}
\esempio{4 6
  
2 4 10 7
  
0 1 1

1 2 3

1 3 2

3 2 2

3 0 0

0 3 4}{16}

No comment.



\vspace{0.5cm}
\esempio{10 9

  3 8 9 6 9 1 6 7 7 9

3 4 1

0 1 2

4 0 4

5 0 1

8 0 0

8 2 1

1 8 2

7 9 1

6 7 2}{54}

Dove $54$ è il valore di $h=[2, 4, 7, 6, 7, 1, 6, 7, 6, 8]$.


\section*{Limitazioni}

\begin{itemize}[nolistsep, itemsep=2mm]
	\item $1 \le N, M \le 100\,000$.
	\item $1 \le H_i \le 10^{12}$.
	\item $0 \leq C_j \le 10^9$ per ogni $j$.
	\item $0 \leq A_j, B_j \leq N - 1$ per ogni $j$.
	\item Non ci sono vincoli duplicati (non esistono $j$, $k$ distinti tali che
	      $A_j = A_k$ e $B_j = B_k$).
	\item Non ci sono vincoli di un costruttore verso se stesso (tali che $A_j =
	      B_j$).
\end{itemize}

\section*{Subtask}

  \begin{itemize}
  \item \textbf{Subtask 1 [\phantom{1}0 punti]:} i casi di esempio forniti alla pagina del problema, essi includono i 3 casi sopra.
      \vspace{-0.6cm}
       \begin{center}
      \rule{0.5\textwidth}{0.4pt} \hfill  \hfill \hfill
       \end{center}
      \vspace{-0.6cm}      
    \item \textbf{Subtask 2 [\phantom{1}4 punti]:} $N \leq 5$ e le altezze massime sono tutte $\leq 5$.
    \item \textbf{Subtask 3 [\phantom{1}8 punti]:} $M = N - 1$ e $B_j = A_j + 1$ per ogni $j$.
    \item \textbf{Subtask 4 [\phantom{1}8 punti]:} $B_j > A_j$.
    \item \textbf{Subtask 5 [20 punti]:} $N \leq 300$.
    \item \textbf{Subtask 6 [16 punti]:} $C_j$ vale $0$ o $1$, per ogni $j$.
    \item \textbf{Subtask 7 [24 punti]:} $N \leq 2000, M \leq 10000$.
    \item \textbf{Subtask 8 [20 punti]:} Nessuna limitazione specifica.
  \end{itemize}
