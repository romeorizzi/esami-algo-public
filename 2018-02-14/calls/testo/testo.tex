\documentclass[a4paper,11pt]{article}
\usepackage{nopageno} % visto che in questo caso abbiamo una pagina sola
\usepackage{lmodern}
\renewcommand*\familydefault{\sfdefault}
\usepackage{sfmath}
\usepackage[utf8]{inputenc}
\usepackage[T1]{fontenc}
\usepackage[italian]{babel}
\usepackage{indentfirst}
\usepackage{graphicx}
\usepackage{tikz}
\usepackage{wrapfig}
\newcommand*\circled[1]{\tikz[baseline=(char.base)]{
		\node[shape=circle,draw,inner sep=2pt] (char) {#1};}}
\usepackage{enumitem}
% \usepackage[group-separator={\,}]{siunitx}
\usepackage[left=2cm, right=2cm, bottom=2cm]{geometry}
\frenchspacing

\newcommand{\num}[1]{#1}

% Macro varie...
\newcommand{\file}[1]{\texttt{#1}}
\renewcommand{\arraystretch}{1.3}
\newcommand{\esempio}[2]{
\noindent\begin{minipage}{\textwidth}
\begin{tabular}{|p{11cm}|p{5cm}|}
	\hline
	\textbf{File \file{input.txt}} & \textbf{File \file{output.txt}}\\
	\hline
	\tt \small #1 &
	\tt \small #2 \\
	\hline
\end{tabular}
\end{minipage}
}

\newcommand{\sezionetesto}[1]{
    \section*{#1}
}

\newcommand{\gara}{Esame algoritmi 2018-02-14 VR}

%%%%% I seguenti campi verranno sovrascritti dall'\include{nomebreve} %%%%%
\newcommand{\nomebreve}{}
\newcommand{\titolo}{}

% Modificare a proprio piacimento:
\newcommand{\introduzione}{
%    \noindent{\Large \gara{}}
%    \vspace{0.5cm}
    \noindent{\Huge \textbf \titolo{}~(\texttt{\nomebreve{}})}
    \vspace{0.2cm}\\
}

\begin{document}

\renewcommand{\nomebreve}{calls}
\renewcommand{\titolo}{Conta le chiamate}

\introduzione{}

Questo esercizio ti chiede di scrivere una versione accellerata
del seguente programma che trovi nella cartella att:
\begin{verbatim}
#include <cstdio>
#include <cassert>

int tot = 0;

void f(int a, int b, int c) {
  tot = (tot+1)%1024;
  if(a>b) {
    f(a-1, b, c/2+1);
    f(a-1, b, c/2);
    f(a-5, b+5, c+1);
  }
}

int main() {
  #ifdef EVAL
      assert( freopen("input.txt", "r", stdin) );
      assert( freopen("output.txt", "w", stdout) );
  #endif

  int a, b, c;
  scanf("%d%d%d", &a, &b, &c);
  f(a,b,c);
  printf("%d\n", tot);
  return 0;
}
\end{verbatim}

Nota che il programma restituisce in output il numero di chiamate della funzione ricorsiva $f$, reso modulo $1024$ per evitare ogni possibile overflow.

\sezionetesto{Dati di input}
La prima e sola riga del file \verb'input.txt' contiene i numeri interi e positivi $a$, $b$, $c$ separati da spazi.
Quando EVAL \`e definito tali parametri vengono letti da standard input invece che da file.

\sezionetesto{Dati di output}
Nel file \verb'output.txt' si scriva un'unica riga contenente un solo numero intero.
Quando EVAL \`e definito tale numero viene scritto su standard output invece che su file.\\


Devi scrivere un programma equivalente, nel senso che rispetti le stesse specifiche input/output,
ma termini in tempo utile.\\

% Assunzioni
\sezionetesto{Assunzioni e note}
\begin{itemize}[nolistsep, noitemsep]
\item $0 \leq a,b,c \leq 100\,000\,000$.
\end{itemize}
  
  \section*{Subtask}
  \begin{itemize}
    \item \textbf{Subtask 1 [0 punti]:} $a \leq 25$.
    \item \textbf{Subtask 2 [20 punti]:} $a \leq 85$.
    \item \textbf{Subtask 3 [20 punti]:} $a \leq 1000$.
    \item \textbf{Subtask 4 [30 punti]:} $90\,000 \leq a,b \leq 100\,000$.
    \item \textbf{Subtask 5 [30 punti]:} $a \leq 200\,000$.
  \end{itemize}
  


\end{document}
