\renewcommand{\nomebreve}{collimator}
\renewcommand{\titolo}{Collimatore a foglia per radioterapia}

\introduzione{}
Ci viene fornito un vettore $rad[1..n]$ di $n$ numeri naturali.
Per ogni coppia di naturali $a$ e $b$, con $1\leq a \leq b \leq n$,
risulta definito il \emph{vettore intervallo} $L_{a,b}[1..n]$ con
\[
L_{a,b}[i] = \left\{
               \begin{array}{ll}
                 1 & \mbox{se $a\leq i \leq b$,}\\
                 0 & \mbox{altrimenti.}
               \end{array}
            \right.
\]

\`E sempre possibile trovare dei moltiplicatori $\lambda_{a,b}$           
 tali che $rad = \sum_{a,b} \lambda_{a,b} L_{a,b}$.
 Di fatto esiste una soluzione che impiega al pi\`u $n$ moltiplicatori non nulli, dato che $rad = \sum_{a=1}^n rad[a] L_{a,a}$.
            
  Allo scopo di ridurre i tempi delle sedute, per medico e paziente, trova una soluzione nella quale i valori $\lambda_{a,b} > 0$ siano nel minor numero possibile.\\


\sezionetesto{Dati di input}
La prima riga del file \verb'input.txt' contiene un numero intero e positivo $n$.
La seconda riga offre una sequenza di $n$ numeri interi separati da spazio:
l'$i$-esimo di questi numeri riporta il valore $rad[i]$.

\sezionetesto{Dati di output}
Nell'unica riga del file \verb'output.txt' si scriva un'unico numero:
il minimo numero di moltiplicatori non nulli che ti consentano di ottenere $rad$
come combinazione lineare intera dei vettori $L_{a,b}$.\\


% Esempi
\sezionetesto{Esempio di input/output}
\esempio{
6

1 2 1 5 1 2
}{4}
\esempio{
5

1 3 6 4 1
}{4}

% Assunzioni
\sezionetesto{Assunzioni e note}
\begin{itemize}[nolistsep, noitemsep]
  \item $1 \le n \le 1\,000\,000$.
\end{itemize}
  
  \section*{Subtask}
  \begin{itemize}
    \item \textbf{Subtask 1 [0 punti]:} i due esempi del testo.
    \item \textbf{Subtask 2 [20 punti]:} $rad[0]=rad[n]=0$ e $|rad[i]-rad[j]|\leq |i-j|$ per ogni $i,j \leq n \leq 1000$.
    \item \textbf{Subtask 3 [10 punti]:} $rad[0]=rad[n]=0$ e $|rad[i]-rad[j]|\leq |i-j|$ per ogni $i,j \leq n \leq 100\,000$.
    \item \textbf{Subtask 4 [10 punti]:} $rad[0]=rad[n]=0$ e $|rad[i]-rad[j]|\leq |i-j|$ per ogni $i,j$.
    \item \textbf{Subtask 5 [10 punti]:} $n \leq 20$.
    \item \textbf{Subtask 6 [10 punti]:} $n \leq 100$.
    \item \textbf{Subtask 7 [10 punti]:} $n \leq 1000$.
    \item \textbf{Subtask 8 [10 punti]:} $n \leq 50000$.
    \item \textbf{Subtask 8 [20 punti]:} nessuna restrizione.
  \end{itemize}
  
