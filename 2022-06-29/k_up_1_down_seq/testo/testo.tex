\documentclass[a4paper,11pt]{article}
\usepackage{nopageno} % visto che in questo caso abbiamo una pagina sola
\usepackage{lmodern}
\renewcommand*\familydefault{\sfdefault}
\usepackage{sfmath}
\usepackage[utf8]{inputenc}
\usepackage[T1]{fontenc}
\usepackage[italian]{babel}
\usepackage{indentfirst}
\usepackage{graphicx}
\usepackage{animate}
\usepackage{tikz}
\usepackage{wrapfig}
\newcommand*\circled[1]{\tikz[baseline=(char.base)]{
		\node[shape=circle,draw,inner sep=2pt] (char) {#1};}}
\usepackage{enumitem}
% \usepackage[group-separator={\,}]{siunitx}
\usepackage[left=2cm, right=2cm, bottom=2cm]{geometry}
\frenchspacing

\newcommand{\num}[1]{#1}

% Macro varie...
\newcommand{\file}[1]{\texttt{#1}}
\renewcommand{\arraystretch}{1.3}
\newcommand{\esempio}[2]{
\noindent\begin{minipage}{\textwidth}
\begin{tabular}{|p{5cm}|p{11cm}|}
	\hline
	\textbf{da \file{stdin}} & \textbf{su \file{stdout}}\\
	\hline
	\tt \small #1 &
	\tt \small #2 \\
	\hline
\end{tabular}
\end{minipage}
}

\newcommand{\sezionetesto}[1]{
    \section*{#1}
}

\newcommand{\gara}{Esame algoritmi 2021-09-17 VR}

%%%%% I seguenti campi verranno sovrascritti dall'\include{nomebreve} %%%%%
\newcommand{\nomebreve}{}
\newcommand{\titolo}{}

% Modificare a proprio piacimento:
\newcommand{\introduzione}{
%    \noindent{\Large \gara{}}
%    \vspace{0.5cm}
    \noindent{\Huge \textbf \titolo{}~(\texttt{\nomebreve{}})}
    \vspace{0.2cm}\\
}

\begin{document}

\renewcommand{\nomebreve}{k\_up\_1\_down\_seq}
\renewcommand{\titolo}{Count the possible sequences with uprisings of size k and downfalls of 1\\}

\introduzione{}

Per via di nostre analisi sui mercati finanziari, siamo interessati nelle $n,k$\emph{-sequenze}, ossia in quelle sequenze di numeri naturali $x_1, x_2, \ldots, x_n$
con $x_1=x_n=0$ e dove, per ogni $i=2,\ldots, n$, si abbia $x_i \geq 0$
e $x_i \in \{x_{i-1}, x_{i-1}-1, x_{i-1}+k\}$.
In altre parole, ogni elemento della sequenza è un numero naturale pari al precedente, oppure al precedente ridotto di $1$, oppure al precedente incrementato di $k$.\\ 

Indichiamo con $f(n,k)$ il numero di $n,k$-sequenze, per un certo $n\geq 1$ e per un certo $k\geq 1$. Per $n=5$, le seguenti tabelle riportanto tutte le possibili $5,k$-sequenze per $k=1$ e per $k=2$, disposte in \emph{ordine lessicografico}.

\begin{table}[!h]
  \begin{minipage}{.5\textwidth}
    \centering
    \begin{tabular}{l|r}
      \textbf{sequence} & \textbf{rank} \\
      \hline
      0 0 0 0 0 & 0 \\
      0 0 0 1 0 & 1 \\
      0 0 1 0 0 & 2 \\
      0 0 1 1 0 & 3 \\
      0 1 0 0 0 & 4 \\
      0 1 0 1 0 & 5 \\
      0 1 1 0 0 & 6 \\
      0 1 1 1 0 & 7 \\
      0 1 2 1 0 & 8 \\
    \end{tabular}
    \caption{lista delle $5,1$-sequenze}
  \end{minipage}%
  \begin{minipage}{.5\textwidth}
    \centering
    \begin{tabular}{l|r}
      \textbf{sequence} & \textbf{rank} \\
      \hline
      0 0 0 0 0 & 0 \\
      0 0 2 1 0 & 1 \\
      0 2 1 0 0 & 2 \\
      0 2 1 1 0 & 3 \\
      0 2 2 1 0 & 4 \\
    \end{tabular}
    \caption{lista delle $5,2$-sequenze}
    \end{minipage}
%}      
\end{table}

Proponiamo di ideare e realizzare procedure che attengano i seguenti goal:

\begin{description}
  \item[goal~1] dati $n$ e $k$, calcolare il valore $f(n,k) \% 1\,000\,000\,007$, ossia il resto della divisione di $f(n,k)$ per il numero $1\,000\,000\,007$
  
  \item[goal~2] dati in ingresso $n$, $k$, $r$, produrre la $n,k$-sequenza di rango $r$. (Assegneremo valori di $n$, $k$ tali da assicurare che le $n,k$-sequenze siano meno di $1\,000\,000\,007$.)
  
  \item[goal~3] data in ingresso una $n,k$-sequenza, calcolare il rango della sequenza data, ossia la sua posizione (partendo da $0$) nella lista delle $n,k$-sequenze disposte secondo l'ordinamento lessicografico. (Si calcoli il rango modulo $1\,000\,000\,007$; esso è univocamente definito anche ove le $n,k$-sequenze siano oltre questo tetto.)
  
  \item[goal~4] dati in ingresso $n$, $k$, listare una per riga, ed in ordine lessicografico, tutte le $n,k$-sequenze
    
\end{description}


\sezionetesto{Input ed Output}

Input ed output avvengono da \verb'stdin' e su \verb'stdout' rispettivamente.
La prima riga dell'input contiene i tre numeri naturali $g$, $n$ e $k$, nell'ordine e separati da spazio. Il parametro $g\in \{1,2,3,4\}$ specifica il goal. Il parametro $n$ indica la lunghezza delle sequenze di interesse, ossia il loro numero di elementi. Il parametro $k$ indica l'entità di quanto possa eventualmente crescere un elemento rispetto a quello che lo precede immediatamente nella sequenza.
Se $g=1$ o $g=4$ l'input consta di questa sola riga.
Se $g=2$ segue una riga contenente il solo numero intero $r$.
Se $g=3$ segue una riga contenente $n$ numeri naturali separati da spazio, quelli di una $n,k$-sequenza di cui si chiede di calcolare il rango.

\sezionetesto{Organizzazione del lavoro}

Suggerimento 1: questo problema è un esercizio sull'approccio ricorsivo alla soluzione dei problemi, come alla programmazione.

Suggerimento 2: il goal~1 è da affrontare per primo e propedeutico ai goal~2,3. I goal~2,3 sono tra di loro complementari oltre che l'inverso uno dell'altro. Per il goal~4 potrai avvalerti della procedura realizzata per il goal~2.

Suggerimento 3: per non limitarsi a risolvere solo casi minuscoli, della ricorsione individuata andrà data un'implementazione all'insegna dell'efficienza.


% Esempi
\sezionetesto{Esempio di input/output}

In attachment alla pagina del problema trovate diverse coppie input/output tra cui le seguenti.

\vspace{0.5cm}
\esempio{1 5 1}{9}

\vspace{0.5cm}
\esempio{1 5 2}{5}

\vspace{0.5cm}
\esempio{2 5 2

3}{0 2 1 1 0}

\vspace{0.5cm}
\esempio{3 5 2
  
0 2 1 1 0}{3}

\vspace{0.5cm}
\esempio{4 5 2}{0 0 0 0 0

0 0 2 1 0

0 2 1 0 0

0 2 1 1 0

0 2 2 1 0}

\newpage

\section*{Subtask}

  \begin{itemize}
  \item \textbf{Subtask 1 [0 punti]:} i casi di esempio forniti alla pagina del problema, essi includono i 5 casi sopra.
      \vspace{-0.6cm}
       \begin{center}
      \rule{0.5\textwidth}{0.4pt} \hfill  \hfill \hfill
       \end{center}
      \vspace{-0.6cm}      
    \item \textbf{Subtask 2 [7 punti]:} $g=1$, $k=1$, $n \le 16$.
    \item \textbf{Subtask 3 [7 punti]:} $g=1$, $n \le 16$.
    \item \textbf{Subtask 4 [5 punti]:} $g=1$, $k=1$, $n \le 100$.
    \item \textbf{Subtask 5 [5 punti]:} $g=1$, $n \le 100$.
    \item \textbf{Subtask 6 [3 punti]:} $g=1$, $k=1$, $n \le 1000$.
    \item \textbf{Subtask 7 [3 punti]:} $g=1$, $n \le 1000$.
      \vspace{-0.6cm}
       \begin{center}
      \rule{0.5\textwidth}{0.4pt} \hfill  \hfill \hfill
       \end{center}
      \vspace{-0.6cm}      

    \item \textbf{Subtask 8 [7 punti]:} $g=2$, $k=1$, $n \le 16$.
    \item \textbf{Subtask 9 [7 punti]:} $g=2$, $n \le 16$.
    \item \textbf{Subtask 10 [5 punti]:} $g=2$, $k=1$, $n \le 23$. (Dove le $n,k$-sequenze sono meno di $1\,000\,000\,007$.)
    \item \textbf{Subtask 11 [5 punti]:} $g=2$, $k>1$, $n \le 25$. (Dove le $n,k$-sequenze sono meno di $1\,000\,000\,007$.)
      \vspace{-0.6cm}
       \begin{center}
      \rule{0.5\textwidth}{0.4pt} \hfill  \hfill \hfill
       \end{center}
      \vspace{-0.6cm}      

    \item \textbf{Subtask 12 [5 punti]:} $g=3$, $k=1$, $n \le 16$.
    \item \textbf{Subtask 13 [5 punti]:} $g=3$, $n \le 16$.
    \item \textbf{Subtask 14 [5 punti]:} $g=3$, $k=1$, $n \le 100$.
    \item \textbf{Subtask 15 [5 punti]:} $g=3$, $n \le 100$.
    \item \textbf{Subtask 16 [3 punti]:} $g=3$, $k=1$, $n \le 1000$.
    \item \textbf{Subtask 17 [3 punti]:} $g=3$, $n \le 1000$.
      \vspace{-0.6cm}
       \begin{center}
      \rule{0.5\textwidth}{0.4pt} \hfill  \hfill \hfill
       \end{center}
      \vspace{-0.6cm}      

    \item \textbf{Subtask 18 [10 punti]:} $g=4$, $k=1$, $n \le 16$.
    \item \textbf{Subtask 19 [10 punti]:} $g=4$, $n \le 16$.
      
  \end{itemize}
  


\end{document}
