% Template per generare 

\documentclass[a4paper,11pt]{article}
\usepackage{lmodern}
\renewcommand*\familydefault{\sfdefault}
\usepackage[utf8]{inputenc}
\usepackage[T1]{fontenc}
\usepackage[italian]{babel}
\usepackage{indentfirst}
\usepackage{graphicx}
\usepackage{tikz}
\newcommand*\circled[1]{\tikz[baseline=(char.base)]{
		\node[shape=circle,draw,inner sep=2pt] (char) {#1};}}
% \usepackage[group-separator={\,}]{siunitx}
\usepackage[left=2cm, right=2cm, bottom=3cm]{geometry}
\frenchspacing

\newcommand{\num}[1]{#1}

% Macro varie...
\newcommand{\file}[1]{\texttt{#1}}
\renewcommand{\arraystretch}{1.3}
\newcommand{\esempio}[2]{
\noindent\begin{minipage}{\textwidth}
\begin{tabular}{|p{11cm}|p{5cm}|}
	\hline
	\textbf{da \file{stdin}} & \textbf{su \file{stdout}}\\
	\hline
	\tt \small #1 &
	\tt \small #2 \\
	\hline
\end{tabular}
\end{minipage}
}

% Dati del task
\newcommand{\gara}{appello algoritmi 2022-06-29}
\newcommand{\nome}{A-sequenze di lunghezza massima}
\newcommand{\nomebreve}{A\_seq}

\begin{document}
% Intestazione
\noindent{\Large \gara}
\vspace{0.5cm}

\noindent{\Huge \textbf \nome~(\texttt{\nomebreve})}

% Descrizione del task
\section*{Descrizione del problema}

Riceverai in input una sequenza $s=s_1,s_2, \ldots, s_n$ di $n$ numeri naturali.
Un sottoinsieme $I$ dell'insieme degli indici $\{1,2, \ldots, n\}$ definisce la \emph{sottosequenza} di $s$ ottenuta da $s$ per rimozione di quegli elementi i cui indici non sono contenuti in $I$.
Ad esempio, quando $n\geq 8$ e con $I=\{3,5,8\}$, otteniamo la sottosequenza $s[I]=s_3,s_5,s_8$, che possiamo indicare anche con $s[3,5,8]$.

Data una sequenza, siamo interessati ad individuare le sue sottosequenze di lunghezza massima, tra quelle con una certa proprietà.
In questo esercizio ci interessano le A-sequenze.
Una sequenza è detta una \emph{A-sequenza} se cresce strettamente fino ad un certo punto, e da quì in poi decresce strettamente.
In altre parole, $a_1,a_2,\ldots, a_{n-1},a_n$ è una \emph{A-sequenza} se e solo se esiste un $\overline{\imath}\in [1,n]$ tale che le seguenti due condizioni valgono entrambe:
\begin{itemize}
  \item[1.] la sequenza $a_1,a_2,\ldots, a_{\overline{\imath}}$ è strettamente crescente;
  \item[2.] la sequenza $a_{\overline{\imath}}, a_{\overline{\imath}+1},\ldots, a_n$ è strettamente decrescente.
\end{itemize}

Proponiamo di ideare e realizzare procedure che attengano i seguenti goal:

\begin{description}
  \item[goal~1] data in input una sequenza $s$, calcolare la massima lunghezza di una sua A-sottosequenza.
  
  \item[goal~2] data in input una sequenza $s$ di lunghezza $n$, sia $f(s)$ il numero di quegli insiemi $I\subseteq \{1,2, \ldots, n\}$ di massima cardinalità tali che $s[I]$ è una A-sequenza. Calcolare il valore $f(s) \% 1\,000\,000\,007$, ossia il resto della divisione di $f(s)$ per il numero $1\,000\,000\,007$.
    
\end{description}

\section*{Input ed Output}
Input ed output avvengono da \verb'stdin' e su \verb'stdout' rispettivamente.

La prima riga dell'input contiene i due numeri naturali $g$ ed $n$, nell'ordine e separati da spazio. Il parametro $g\in \{1,2\}$ specifica il goal. Il parametro $n$ indica la lunghezza della sequenza $s$, ossia il suo numero di elementi. La seconda e ultima riga offre gli $n$ elementi di $s$, nel loro ordine e separati da spazi.

L'output consta di un solo numero naturale:
la massima lunghezza di una A-sottosequenza di $s$ quando $g=1$, altrimenti il numero (preso modulo $1\,000\,000\,007$) di quegli insiemi $I\subseteq \{1,2, \ldots, n\}$ tali che $s[I]$ è una A-sottosequenza di $s$ di massima lunghezza.

% Assunzioni
\section*{Assunzioni}

\begin{itemize}
\item $1 \le n \le 10\,000$
\item $0 \le s_i < 100\,000$
\end{itemize}

\newpage
\section*{Subtask}
\begin{itemize}
  \item \textbf{Subtask 1 [0 punti]:} i casi di esempio forniti alla pagina del problema, essi includono i 4 casi analizzati sotto.
      \vspace{-0.6cm}
       \begin{center}
      \rule{0.5\textwidth}{0.4pt} \hfill  \hfill \hfill
       \end{center}
      \vspace{-0.6cm}      
    \item \textbf{Subtask 2 [13 punti]:} $g=1$, $n \le 20$.
    \item \textbf{Subtask 3 [10 punti]:} $g=1$, $n \le 100$.
    \item \textbf{Subtask 4 [10 punti]:} $g=1$, $n \le 1000$.
    \item \textbf{Subtask 5 [15 punti]:} $g=1$, $n \le 10\,000$.
      \vspace{-0.6cm}
       \begin{center}
      \rule{0.5\textwidth}{0.4pt} \hfill  \hfill \hfill
       \end{center}
      \vspace{-0.6cm}      
    \item \textbf{Subtask 6 [10 punti]:} $g=2$, $n \le 20$.
    \item \textbf{Subtask 7 [15 punti]:} $g=2$, $n \le 100$.
    \item \textbf{Subtask 8 [15 punti]:} $g=2$, $n \le 1000$.
    \item \textbf{Subtask 9 [12 punti]:} $g=2$, $n \le 10\,000$.
\end{itemize}

\section*{Esempi di input/output}
\esempio{
1 8

0 9 8 5 1 8 4 7
}{5}

\esempio{
2 8

0 9 8 5 1 8 4 7
}{2}

\esempio{
1 6

3 3 3 3 3 3}{1}

\esempio{
2 6

3 3 3 3 3 3}{6}

\section*{Spiegazioni}
Nel primo esempio, la risposta corretta è~$5$ in quanto $s[1,2,3,4,5]=0,9,8,5,1$ è una A-sottosequenza di $s$ di lunghezza~$5$, e non ne vedo di più lunghe.

Nel secondo esempio, la risposta corretta è~$2$ in quanto $s[1,2,3,4,7]$ è un'altra A-sottosequenza di $s$ di lunghezza~$5$, e non ve ne sono altre.

Nel terzo esempio, con gli elementi di $s$ tutti uguali e considerato che nessuna A-sequenza può contenere due elementi consecutivi uguali, allora~$1$ è la massima lunghezza di una A-sottosequenza di $s$.

Nel quarto esempio, le A-sottosequenze di $s$ di lunghezza massima (ossia di lunghezza $1$) sono queste sei: $s[1]$, $s[2]$, $s[3]$, $s[4]$, $s[5]$, $s[6]$.

\end{document}
