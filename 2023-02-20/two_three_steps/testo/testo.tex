\documentclass[a4paper,11pt]{article}
\usepackage{nopageno} % visto che in questo caso abbiamo una pagina sola
\usepackage{lmodern}
\renewcommand*\familydefault{\sfdefault}
\usepackage{sfmath}
\usepackage[utf8]{inputenc}
\usepackage[T1]{fontenc}
\usepackage[italian]{babel}
\usepackage{indentfirst}
\usepackage{graphicx}
\usepackage{tikz}
\usepackage{wrapfig}
\newcommand*\circled[1]{\tikz[baseline=(char.base)]{
		\node[shape=circle,draw,inner sep=2pt] (char) {#1};}}
\usepackage{enumitem}
% \usepackage[group-separator={\,}]{siunitx}
\usepackage[left=2cm, right=2cm, bottom=2cm]{geometry}
\frenchspacing

\newcommand{\num}[1]{#1}

% Macro varie...
\newcommand{\file}[1]{\texttt{#1}}
\renewcommand{\arraystretch}{1.3}
\newcommand{\esempio}[2]{
\noindent\begin{minipage}{\textwidth}
\begin{tabular}{|p{11cm}|p{5cm}|}
	\hline
	\textbf{File \file{input.txt}} & \textbf{File \file{output.txt}}\\
	\hline
	\tt \small #1 &
	\tt \small #2 \\
	\hline
\end{tabular}
\end{minipage}
}

\newcommand{\sezionetesto}[1]{
    \section*{#1}
}

\newcommand{\gara}{Esame algoritmi 2023-02-20 VR}

%%%%% I seguenti campi verranno sovrascritti dall'\include{nomebreve} %%%%%
\newcommand{\nomebreve}{}
\newcommand{\titolo}{}

% Modificare a proprio piacimento:
\newcommand{\introduzione}{
%    \noindent{\Large \gara{}}
%    \vspace{0.5cm}
    \noindent{\Huge \textbf \titolo{}~(\texttt{\nomebreve{}})}
    \vspace{0.2cm}\\
}

\begin{document}

\renewcommand{\nomebreve}{two\_three\_steps}
\renewcommand{\titolo}{Better two or three steps?}

\introduzione{}

Nella nostra passeggiata lungo il corridoio del reparto geriatria,
facciamo pausa ogni 2 oppure 3 piastrelle, e lì ripeschiamo un ricordo felice dalla nostra memoria, aiutati dalle foto di paesaggi appese per nascondere la realtà delle misere pareti.
E però il valore di quel ricordo ci ostiniamo a voler credere sia dettato dai luoghi nel poster, non da noi stessi e nel nostro vissuto, e potremmo essere indotti a ritenere di voler programmare le nostre pause in un corridoio rappresentabile come un array di numeri naturali, che noi si percorre avanzando di due o tre posizioni per volta, e aggiungendo il numero ivi riscontrato, ad ogni sosta, al senso ultimo della nostra vita di follia.


  Si massimizzi il senso di tutto questo.\\


\sezionetesto{Dati di input}
L'input avviene da \verb'stdin'. La prima riga contiene un numero intero e positivo $n$, il numero di poster appesi lungo il corridoio.
La seconda riga offre una sequenza di $n$ numeri naturali separati da spazio:
l'$i$-esimo di questi numeri, $GGG_i$, esprime la qualità del sogno ispiratomi da quella fotografia. E ovviamente i numeri sono sommabili, no?

\sezionetesto{Dati di output}
L'input avviene su \verb'stdout'. Nella prima ed unica riga si scriva il massimo valore per la somma delle qualità dei sogni di cui fruirò.\\


% Esempi
\sezionetesto{Esempio di input/output}
\esempio{
6

0 1 3 3 0 0
}{3}
\esempio{
6

0 0 2 3 1 1
}{4}

{\bf spiegazione.} In entrambi i casi il poster da cui si parte, ossia quella disposta a inizio corridoio, vale~$0$ e non offre pertanto alcun contributo alla somma.
Nel primo caso il secondo poster su cui si sosta vale~$3$, che ci si sposti di due o di tre posizioni. In entrambi i casi stazioneremo poi di fronte ad un terzo ed ultimo poster, che comunque sia non aggiungerà alcun valore.
Nel secondo caso il secondo poster su cui si sosta vale o~$2$ o~$3$, e il terzo vale poi comunque 1.
In entrambi i casi il numero di poster visionati sarà sempre~3, comunque si gestisca la camminata lungo il corridoio di geriatria.


\section*{Subtask}

  \begin{itemize}
    \item \textbf{Subtask 1 [0 punti]:} i due esempi del testo.
    \item \textbf{Subtask 2 [25 punti]:} $n \leq 10$.
    \item \textbf{Subtask 3 [25 punti]:} $n \leq 100$.
    \item \textbf{Subtask 4 [30 punti]:} $n \leq 1\,000$.
    \item \textbf{Subtask 5 [20 punti]:} $n \leq 1\,000\,000$.
  \end{itemize}
  


\end{document}
