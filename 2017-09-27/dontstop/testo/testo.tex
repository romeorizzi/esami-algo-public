% Template per generare

\documentclass[a4paper,11pt]{article}
\usepackage{lmodern}
\renewcommand*\familydefault{\sfdefault}
\usepackage{sfmath}
\usepackage[utf8]{inputenc}
\usepackage[T1]{fontenc}
\usepackage[italian]{babel}
\usepackage{indentfirst}
\usepackage{graphicx}
\usepackage{tikz}
\usetikzlibrary{graphs,arrows.meta}
\newcommand*\circled[1]{\tikz[baseline=(char.base)]{
    \node[shape=circle,draw,inner sep=2pt] (char) {#1};}}
\usepackage{enumitem}
% \usepackage[group-separator={\,}]{siunitx}
\usepackage[left=2cm, right=2cm, bottom=3cm]{geometry}
\frenchspacing

\newcommand{\num}[1]{#1}

% Macro varie...
\newcommand{\file}[1]{\texttt{#1}}
\renewcommand{\arraystretch}{1.3}
\newcommand{\esempio}[2]{
  \noindent\begin{minipage}{\textwidth}
    \begin{tabular}{|p{11cm}|p{5cm}|}
      \hline
      \textbf{File \file{input.txt}} & \textbf{File \file{output.txt}}\\
      \hline
      \tt \small #1 &
      \tt \small #2 \\
      \hline
    \end{tabular}
  \end{minipage}
}

% Dati del task
\newcommand{\gara}{.}
\newcommand{\nome}{Chi si ferma \`e perduto}
\newcommand{\nomebreve}{dontstop}

\begin{document}


  % Intestazione
  \noindent{\Large \gara}
  \vspace{0.5cm}

  \noindent{\Huge \textbf \nome~(\texttt{\nomebreve})}
  \vspace{0.2cm}\\

  % Descrizione del task
  \section*{Descrizione del problema}

  \noindent
  Due giocatori si alternano nello spostare una pedina da un nodo ad un altro di un grafo diretto. Nei turni pari (dispari), il giocatore $P$ (il giocatore $D$) guarda in quale nodo del grafo si trovi la pedina, e guarda quali archi escano da quel nodo ed in quali nodi tali archi portino. Se non esce alcun arco, allora il giocatore di turno non pu\`o muovere la pedina e perde, altrimenti sceglie lungo quale arco uscente muovere la pedina e riconsegna all'avversario la situazione di gioco con la pedina collocata nel nodo verso cui porta l'arco scelto.
La cosa che nessuno dei due vuole nel modo pi\`u assoluto \`e perdere, dopodich\`e entrambi i giocatori vorrebbero far perdere l'avversario, cosa che considerano vittoria. Il gioco pu\`o proseguire indefinitivamente, ma, se un giocatore riesce a far perdere l'avversario \`e questa la direzione in cui andr\`a a remare.\\

Scrivi una procedura che stabilisca, per ogni possibile nodo di partenza per la pedina, l'esito della partita ove entrambi i giocatori giochino ottimamente.
I nodi sono numerati da $0$ a $N-1$.
Per ogni nodo $u$, il valore $\texttt{esito}[u]$ denota l'esito di una partita, ottimamente giocata da entrambi i giocatori, che abbia preso avvio con la pedina collocata sul nodo $u$, e vale:
\begin{itemize}
  \item[$+1$] se il giocatore che muove per primo ha una strategia vincente,
  \item[$-1$] se il giocatore che muove per secondo ha una strategia vincente,
  \item[$0$] se entrambi i giocatori hanno una strategia che consente loro di non perdere mai.
\end{itemize}
Dovrai calcolare e stampare $\texttt{esito}[u]$ per ciascun nodo $u = 0,\dots,N-1$.

\section*{Assunzioni}
\begin{itemize}[nolistsep, noitemsep]
    \item $1 \le N \le 100\,000$, $1 \le M \le 1\,000\,000$.
    \item non vi sono archi ripetuti (ma è possibile avere archi in entrambe le direzioni fra due nodi).
\end{itemize}

\section*{Dati di input}
Ogni riga del file \verb'input.txt' contiene due numeri separati da spazio.  
L'ordine tra i due numeri \`e importante:
nella prima riga troviamo prima $N$ e poi $M$, e le successive $M$ righe contengono gli archi del grafo diretto.
Nello specifico, la riga $1+i$, per $i=0,\dots,M-1$, contiene due numeri $u_i$ e $v_i$ che denotano un arco da $u_i$ a $v_i$.

\section*{Dati di output}
Nel file \verb'output.txt' si scriva un unica riga contenente gli $N$ numeri $\texttt{esito}[u]$, per $u = 0,\dots,N-1$, cos\`\i\ ordinati sull'indice $u$ e separati da spazio.

% Esempi
\section*{Esempio di input/output}

\esempio{
8 11

0 1

2 0

1 2

1 3

2 3

3 4

5 3

5 4

4 6

6 5

6 7
}{
0 0 0 1 -1 1 1 -1

}

Il file di esempio fa riferimento al seguente grafo.
Il valore $\texttt{esito}[u]$ è riportato in prossimità di ciascun nodo $u$.

\begin{center}
    \begin{tikzpicture}[x=20mm,y=12mm,>={Stealth[length=3mm,width=3mm]}]
        \graph[
            no placement,
            nodes={circle,draw,very thick,minimum size=5mm,every label/.style={font=\small}},
            edges={very thick},
        ] {
            0 [x=0, y=0,label={        left:$ 0$}];
            1 [x=1,y=+1,label={  above left:$ 0$}];
            2 [x=1,y=-1,label={  below left:$ 0$}];
            3 [x=2, y=0,label={       above:$+1$}];
            4 [x=3,y=+1,label={ above right:$-1$}];
            5 [x=3,y=-1,label={ below right:$+1$}];
            6 [x=4, y=0,label={       above:$+1$}];
            7 [x=5, y=0,label={       right:$-1$}];

            0 -> 1;
            2 -> 0;
            1 -> 2;
            1 -> 3;
            2 -> 3;
            3 -> 4;
            5 -> 3;
            5 -> 4;
            4 -> 6;
            6 -> 5;
            6 -> 7;
        };
    \end{tikzpicture}
\end{center}

\section*{Subtask}
\begin{itemize}
    \item \textbf{Subtask 0 [\phantom{1}0 punti]:} casi di esempio.
    \item \textbf{Subtask 1 [20 punti]:} $N \le 10$, il grafo è aciclico.
    \item \textbf{Subtask 2 [20 punti]:} $N \le 10$.
    \item \textbf{Subtask 2 [30 punti]:} $N \le 1\,000$.
    \item \textbf{Subtask 2 [30 punti]:} nessuna limitazione specifica.
\end{itemize}

\end{document}
