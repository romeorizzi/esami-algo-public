\documentclass[a4paper,11pt]{article}
\usepackage{nopageno} % visto che in questo caso abbiamo una pagina sola
\usepackage{lmodern}
\renewcommand*\familydefault{\sfdefault}
\usepackage{sfmath}
\usepackage[utf8]{inputenc}
\usepackage[T1]{fontenc}
\usepackage[italian]{babel}
\usepackage{indentfirst}
\usepackage{graphicx}
\usepackage{tikz}
\usepackage{wrapfig}
\newcommand*\circled[1]{\tikz[baseline=(char.base)]{
		\node[shape=circle,draw,inner sep=2pt] (char) {#1};}}
\usepackage{enumitem}
% \usepackage[group-separator={\,}]{siunitx}
\usepackage[left=2cm, right=2cm, bottom=2cm]{geometry}
\frenchspacing

\newcommand{\num}[1]{#1}

% Macro varie...
\newcommand{\file}[1]{\texttt{#1}}
\renewcommand{\arraystretch}{1.3}
\newcommand{\esempio}[2]{
\noindent\begin{minipage}{\textwidth}
\begin{tabular}{|p{11cm}|p{5cm}|}
	\hline
	\textbf{File \file{input.txt}} & \textbf{File \file{output.txt}}\\
	\hline
	\tt \small #1 &
	\tt \small #2 \\
	\hline
\end{tabular}
\end{minipage}
}

\newcommand{\sezionetesto}[1]{
    \section*{#1}
}

\newcommand{\gara}{Esame algoritmi 2017-09-31 VR}

%%%%% I seguenti campi verranno sovrascritti dall'\include{nomebreve} %%%%%
\newcommand{\nomebreve}{}
\newcommand{\titolo}{}

% Modificare a proprio piacimento:
\newcommand{\introduzione}{
%    \noindent{\Large \gara{}}
%    \vspace{0.5cm}
    \noindent{\Huge \textbf \titolo{}~(\texttt{\nomebreve{}})}
    \vspace{0.2cm}\\
}

\begin{document}

\renewcommand{\nomebreve}{easyfall}
\renewcommand{\titolo}{Tutti gi\`u per terra}

\introduzione{}

  Sul pavimento \`e disposta una sequanza di $n$ domini, essi sono cos\`\i\ incastrati che ciascuno di loro pu\`o cadere solo verso destra.  
  Per ogni $i=1,\ldots n$,
  il domino~$i$ ha altezza $h_i\in \mathbf{N}\setminus \{0\}$ e, se cade,
  cadranno tutti i domini di indice~$j$, con~$j \in [i,i+h_i[$.
  I domini di altezza~$1$ sono pertanto i soli che possono cadere da soli.  

  Pierino ha $k$ palline per la sua cerbottana.
  Con ciascuna pallina pu\`o provocare la caduta di un domino a sua scelta.
  Specificare il massimo numero di domini di cui pu\`o provocare la caduta
  per ogni valore di $k$.\\


\sezionetesto{Dati di input}
La prima riga del file \verb'input.txt' contiene un numero intero e positivo $n$.
La seconda riga offre una sequenza di $n$ numeri interi separati da spazio:
l'$i$-esimo di questi numeri riporta l'altezza del domino $i$-esimo, come indicizzati da sinistra verso destra.

\sezionetesto{Dati di output}
Nel file \verb'output.txt' si scriva un unica riga contenente $n$ numeri separati da spazio:
il $k$-esimo di questi numeri indica il massimo numero di tessere di domino
che si possa far cadere impiegando al pi\`u $k$ palline.\\


% Esempi
\sezionetesto{Esempio di input/output}
\esempio{
6

1 2 1 1 1 2
}{2 3 4 5 6 6}
\esempio{
5

3 1 1 2 1
}{3 5 5 5 5}

% Assunzioni
\sezionetesto{Assunzioni e note}
\begin{itemize}[nolistsep, noitemsep]
  \item $1 \le n \le 1\,000\,000$.
\end{itemize}
  
  \section*{Subtask}
  \begin{itemize}
    \item \textbf{Subtask 1 [0 punti]:} i due esempi del testo.
    \item \textbf{Subtask 2 [20 punti]:} $n \leq 10$ e tessere alte~$1$ oppure~$2$.
    \item \textbf{Subtask 3 [20 punti]:} $n \leq 100$ e tessere alte~$1$ oppure~$2$.
    \item \textbf{Subtask 4 [20 punti]:} $n \leq 1000$ e tessere alte~$1$ oppure~$2$.
    \item \textbf{Subtask 5 [20 punti]:} tessere alte~$1$ oppure~$2$.
    \item \textbf{Subtask 6 [20 punti]:} nessuna restrizione.
  \end{itemize}
  


\end{document}
